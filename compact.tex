\documentclass[main]{subfiles}

\begin{document}

\section{Компактные операторы}

Исторически, понятие компактности оператора
появилось под названием «вполне непрерывность».
Оператор \( A : E_1 \to E_2 \)
назывался вполне непрерывным, если
под его действием любая слабо сходящаяся последовательность
начинала сходиться по норме.
Современное определение ушло от понятия последовательности
и аналогично определению ограниченного оператора
как оператора переводящего ограниченное множество в ограниченное.

\begin{definition}
  Оператор \( A \in \Linears{E_1, E_2} \)
  называется \emph{компактным}, если
  для любого ограниченного множества \( X \subset E_1 \)
  его образ \( A X \) предкомпактен.
  Множество компактных операторов, действующих
  из \( E_1 \) в \( E_2 \), будем обозначать как
  \( K(E_1, E_2) \) или, если \( E_1 = E_2 = E \),
  \( K(E) \).
\end{definition}

\begin{remark}
  В банаховых пространствах
  предкомпактность эквивалентна вполне ограниченности;
  однако, в определении мы говорим о предкомпактности "---
  это делает его переносимым на топологические линейные пространства.
\end{remark}

\begin{exercise}
  Если \( A \overline{B}(0, 1) \) "--- предкомпакт,
  то \( A \) компактен.
\end{exercise}

\begin{exampleslist}
  \item Если \( \dim E < \infty \), то все операторы из
    \( \Linears{E} \) компактны.
  \item Если \( \dim E = \infty \), то \( I_E \) не является компактным
    в силу некомпактности сферы в бесконечномерных пространствах.
  \item
    Рассмотрим \( E = C[0, 1] \), определим \( A : E \to E \):
    \[
      (Af)(x) = \int_0^x f(t) dt.
    \]
    Благодаря теореме Арцела"--~Асколи достаточно показать, что
    \( A \overline{B}(0, 1) \) ограниченно и обладает
    свойством равностепенной непрерывности.
    Ограниченность оператора очевидна:
    \[
      |(Af)(x)| = |\int_0^x f(t) dt| \le
      \int_0^x |f(t)| dt \le x \cdot ||f|| \le ||f||,
    \]
    а потому \( ||Af|| \le ||f|| \le 1 \).
    Кроме того,
    \[
      |(Af)(x) - (Af)(y)| =
      \left|\int_y^x f(t) dt \right| \le |x - y| \cdot ||f||,
    \]
    а значит, в определении равностепенной непрерывности
    мы можем брать \( \delta = \epsilon \).
    Итак, \( A \) "--- компактен.
\end{exampleslist}

\begin{exercise}
  \( K(E) \) "--- идеал в \( \Linears{E} \),
  т. е. если \( A \in K(E) \) и \( B \in \Linears{E} \),
  то \( AB, BA \in K(E) \).
\end{exercise}

\begin{corollary}
  В бесконечномерном пространстве компактный оператор
  не имеет ограниченного обратного оператора.
\end{corollary}
\begin{proof}
  Пусть \( E \) "--- линейное нормированное пространство, \( \dim E = \infty \), \( A \in K(E) \).
  Тогда если \( A^{-1} \in \Linears{E} \), то
  \( I = A A^{-1} \in K(E) \), что неверно.
\end{proof}

\begin{theorem}
  Пусть \( E_1 \) и \( E_2 \) "---
  линейные нормированные пространства,
  причём \( E_2 \) "--- банахово,
  \( \{ A_n \}_{n=1}^\infty \subset K(E_1, E_2) \)
  и \( A_n \to A \).
  Тогда \( A \) также компактен.
\end{theorem}
\begin{proof}
  По определению предела и нормы,
  \[
    \Forall{\epsilon > 0} \Exists{N(\epsilon)}
    \Forall{n \ge N}
    \Forall{x \in \overline{B}(0, 1)}
    ||A x - A_n x|| \le \epsilon.
  \]
  Зафиксируем \( \epsilon > 0 \)
  и выберем \( n \ge N(\epsilon) \).
  Поскольку \( A_n \) "--- компактный,
  мы можем построить конечную \( \epsilon \)-сеть
  \( \{ y^n_1, \dots, y^n_{m_n} \} \)
  для \( A_n \Cl{B}(0, 1) \).
  Она же будет \( 2\epsilon \)-сетью для
  \( A \Cl{B}(0, 1) \):
  для произвольного \( x \in \Cl{B}(0, 1) \)
  найдётся \( y^n_k \) такой,
  что \( ||A_n x - y^n_k|| \le \epsilon \);
  кроме того, \( ||A x - A_n x|| \le \epsilon \),
  а потому \( ||A x - y^n_k|| \le 2\epsilon \).
\end{proof}

\begin{exercise}
  Докажите, что оператор
  \( A : \ell_2 \to \ell_2 \) заданный
  по правилу
  \[
    (A x)_n = \lambda_n x_n,
  \]
  компактен тогда и только тогда,
  когда \( \lambda_n \to 0 \).
\end{exercise}

\begin{theorem}
  Пусть \( E(\Complex) \) "--- линейное нормированное пространство,
  \( A \in K(E) \).
  Тогда \( \Forall{\lambda \ne 0} \dim \Ker A_\lambda < \infty \)
  (т. е. собственное пространство,
  соответствующее \( \lambda \),
  конечномерно).
\end{theorem}
\begin{proof}
  В силу теоремы~\ref{thm:sphere-compactness},
  достаточно показать предкомпактность
  единичной сферы в \( \Ker A_\lambda \).
  Итак, выберем из неё
  произвольную последовательность
  \( \{ x_n \}_{n=1}^\infty \subset \Ker A_\lambda \),
  \( \Forall{n} ||x_n|| = 1 \);
  необходимо выделить в ней
  сходящуюся подпоследовательность.
  Поскольку \( A \) компактен,
  множество значений последовательности
  \( \{ A x_n \}_{n=1}^\infty \) "--- предкомпактно,
  и тогда у неё найдётся сходящаяся подпоследовательность:
  \( A x_{n_k} \to y \).
  При этом, \( A x_{n_k} = \lambda x_{n_k} \),
  а потому
  \( x_{n_k} \to \frac{1}\lambda y \).
\end{proof}

\begin{theorem}\label{thm:compact-spectrum-bounds}%13.3
  Пусть \( E(\Complex) \) "--- банахово пространство, \( A \in K(E) \).
  Тогда для любого \( \delta > 0 \) вне круга
  \( \{ |\lambda| \le \delta \} \) может быть
  только конечное число собственных значений оператора \( A \).
\end{theorem}
\begin{proof}
  Докажем теорему только для случая
  гильбертова пространства \( E \)
  и самосопряжённого оператора \( A \).

  Предположим противное: существует последовательность
  различных собственных значений \( \{ \lambda_n \}_{n=1}^\infty \)
  таких, что \( |\lambda_n| > \delta \) для любого \( n \).
  Выберем для каждого собственный вектор \( x_n \),
  при том такой, что \( ||x_n|| = 1 \).
  Поскольку \( A \) компактен,
  \( \{ A x_n \} \) должно быть предкомпактным множеством.
  Но, поскольку у самосопряжённого оператора
  собственные вектора соответствующие
  различным собственным значениям
  ортогональны,
  \[
    ||A x_n - A x_m||^2 = ||\lambda_n x_n - \lambda_m x_m||^2 =
    ||\lambda_n x_n||^2 + ||\lambda_m x_m||^2 =
    |\lambda_n|^2 + |\lambda_m|^2 > \delta^2.
  \]
  Таким образом, счётное количество точек
  имеет попарные расстояния большие \( \delta^2 \),
  что противоречит вполне ограниченности и,
  соответственно, предкомпактности.
\end{proof}

\begin{corollary}
  В условиях теоремы, для произвольного \( \delta > 0 \)
  \[
    \sum_{|\lambda| > \delta} \dim \Ker A_\lambda < \infty
  \]
\end{corollary}

\begin{corollary}
  Если \( A \in K(E) \), то \( \sigma_P(A) \) "---
  не более, чем счётное множество.
\end{corollary}

\begin{problem}
  Если \( A \in K(E_1, E_2) \),
  а последовательность \( \{ x_n \} \subset E_1 \)
  слабо сходится к \( x \),
  то \( A x_n \to A x \) (по норме \( E_2 \)).
\end{problem}

%Линал
%Ax = y разрешима \( \oTTo \) y ортогонален всем
%решениям \( A^T z = 0 \) (\( A^* z = 0 \) на бескоординатном языке).

\begin{lemma}
  Пусть \( H \) "--- комплексное гильбертово пространство,
  \( A \in \Linears{H} \) "--- компактный самосопряжённый оператор.
  Тогда если \( \lambda \ne 0 \) "--- точка спектра \( A \),
  то \( \lambda \) "--- собственное значение \( A \).
\end{lemma}
\begin{proof}
  По критерию принадлежности точки спектру самосопряжённого оператора,
  \( \lambda \in \sigma(A) \) \(\oTTo\) 
  существует последовательность \( \{ x_n \} \)
  такая, что \( ||x_n|| = 1 \) и
  \( ||(A - \lambda I) x_n|| \to 0 \).
  \( \{ A x_n \} \) "--- предкомпакт,
  найдётся сходящаяся подпоследовательность:
  \( A x_{n_k} \to y \).
  Кроме того, конечно,
  \( A x_{n_k} - \lambda x_{n_k} \to 0 \);
  поскольку \( \lambda \ne 0 \),
  \( \frac1\lambda A x_{n_k} - x_{n_k} \to 0 \),
  и тогда
  \[
    \lim x_{n_k} = \lim \frac1\lambda A x_{n_k} - \lim (\frac1\lambda A x_{n_k} - x_{n_k}) =
    \frac1\lambda y.
  \]
  Наконец, поскольку \( A \) непрерывен,
  \[
    y = \lim A x_{n_k} = A \lim x_{n_k} = \frac1\lambda A y,
  \]
  т. е. \( A y = \lambda y \).
\end{proof}

\begin{lemma}%3
  Пусть \( H \) "--- комплексное гильбертово пространство,
  \( A \in \Linears{H} \) "--- самосопряжённый оператор,
  а \( M \) "--- подпространство \( H \),
  инвариантное относительно \( A \)
  (т. е. \( AM \subset M \)).
  Тогда \( M^\perp \) также инвариантно
  относительно \( A \).
\end{lemma}
\begin{proof}
  \( y \in M^\perp \), если \( \Forall{x \in M}
  \Inner{x, y} = 0 \).
  Поскольку для произвольного \( x \in M \)
  также и \( Ax \in M \), то
  \[
    \Inner{x, Ay} = \Inner{Ax, y} = 0,
  \]
  т. е. \( Ay \in M^\perp \).
\end{proof}

\begin{lemma}%2
  Пусть \( H \) "--- комплексное гильбертово пространство,
  \( A \in \Linears{H} \) "--- компактный самосопряжённый оператор.
  Если \( \lambda \ne 0 \),
  то \( \Cl{\Img A_\lambda} = \Img A_\lambda \).
\end{lemma}
\begin{proof}
  Из доказательства теоремы~\ref{thm:selfconjugate-complement}
  следует, что \( (\Ker A_\lambda)^\perp = \overline{\Img A_\lambda} \).
  \( \Ker A_\lambda \) "--- инвариантно относительно \( A \),
  а потому и \( \overline{\Img A_\lambda} \) "--- тоже.
  Рассмотрим оператор
  \[ \tilde A = A \bigr|_{\Cl{\Img A_\lambda}};\]
  он также компактный и самосопряжённый.
  Мы отбросили
  \( \Ker A_\lambda \) (за исключением нуля),
  поэтому  \( \lambda \)
  не может быть собственным значением \( \tilde A \).
  Значит, \( \lambda \in \rho(\tilde A) \).
  Тогда, конечно, образ оператора \( \tilde A_\lambda \) равен
  его области определения "--- \( \overline{\Img A_\lambda} \).
  Наконец, поскольку \( \tilde A_\lambda \) "--- просто сужение \( A_\lambda \),
  \[
    \Img A_\lambda \subset
    \overline{\Img A_\lambda} =
    \Img \tilde A_\lambda \subset
    \Img A_\lambda. \qedhere
  \]
\end{proof}

\begin{theorem*}[Альтернатива Фредгольма]
  Пусть \( H \) "--- комплексное гильбертово пространство,
  \( A \in \Linears{H} \) "--- компактный самосопряжённый оператор,
  \( \lambda \in \Complex \), \( \lambda \ne 0 \).
  Тогда либо \( \lambda \) "--- не собственное значение \( A \),
  и уравнение
  \[
    A x = \lambda x + y
  \]
  имеет решение относительно \( x \),
  определённое для любого \( y \in H \) и
  непрерывно зависящее от него,
  либо \( \lambda \) "--- собственное значение \( A \)
  и это уравнение разрешимо
  (не единственным) образом
  в точности для тех \( y \), которые ортогональны
  всем собственным векторам для \( \lambda \).
\end{theorem*}
\begin{proof}
  Утверждение теоремы напрямую следует из доказанных лемм.
\end{proof}

\begin{lemma}\label{thm:ksco-eigenvalue}
  Пусть \( H \) "--- комплексное гильбертово пространство,
  \( A \in \Linears{H} \) "--- ненулевой компактный самосопряжённый оператор.
  Тогда у \( A \) существует ненулевое собственное значение.
\end{lemma}
\begin{proof}
  Поскольку \( A \) "--- самосопряжённый,
  \( 0 \ne ||A|| = \max \{ |m_-|, |m_+| \} \).
  Значит, хотя бы одно из них ненулевое,
  а поскольку оба они принадлежат спектру,
  мы нашли ненулевую точку спектра. Наконец,
  вспомним, что она обязательно будет
  собственным значением.
\end{proof}

\begin{theorem}[Гильберт"--~Шмидт]
  Пусть \( H \) "--- сепарабельное комплексное гильбертово пространство,
  \( \dim H = \infty \), и
  \( A \in \Linears{H} \) "--- компактный самосопряжённый оператор.
  Тогда в \( H \) существует ортонормированный базис
  из собственных векторов оператора \( A \).
\end{theorem}
\begin{proof}
  Для каждого собственного значения \( \lambda \)
  выберем ортонормированный базис в \( \Ker A_\lambda \);
  в силу теоремы~\ref{thm:compact-spectrum-bounds}
  собственных значений не более, чем счётно много,
  а потому мы можем объединить все эти базисы
  в одну последовательность \( e = \{ e_n \}_{n=1}^N \),
  где, возможно, \( N = \infty \).
  Поскольку собственные вектора для разных значений
  ортогональны, \( e \) "--- ортонормированная система.
  Обозначим \( M = \Cl{[e]} \);
  если мы покажем, что \( M = H \) или,
  эквивалентно, \( M^\perp = \{ 0 \} \),
  то \( e \) "--- требуемый ортонормированный базис \( H \).

  Заметим, что \( M \) инвариантно относительно \( A \):
  если \( \lambda_n \) "--- собственное значение,
  которому соответствует \( e_n \), то
  \[
    A (\sum_{n=1}^N \alpha_n e_n) =
    \sum_{n=1}^N \alpha_n A e_n =
    \sum_{n=1}^N (\alpha_n \lambda_n) e_n \in M.
  \]
  Тогда и \( M^\perp \) инвариантно относительно \( A \),
  а значит \( A_0 = A \bigr|_{M^\perp} \) "---
  компактный самосопряжённый оператор из \( \Linears{M^\perp} \).
  Более того, он не имеет собственных значений.
  Значит, вследствие леммы~\ref{thm:ksco-eigenvalue},
  \( A_0 = 0 \).
  Тогда если \( M^\perp \ne \{ 0 \} \), то \( 0 \) "---
  собственное значение \( A_0 \),
  что противоречит его построению.
\end{proof}

%\begin{example}
%  Доказать, что краевая задача
%  \[
%    \begin{cases}
%      y'' + \lambda \sin y = f(x), & x \in (0, 1) \\
%      y(0) = y(1) = 0
%    \end{cases}
%  \]
%  имеет решение для любых \( \lambda \in \Real \)
%  и \( f \in C[0, 1] \).
%\end{example}

\end{document}
