\documentclass[main]{subfiles}

\begin{document}
\section{Сопряжённый оператор} % 11

Пусть задано два линейных нормированных пространства
\( E_1 \) и $E_2$;
пусть также зафиксирован оператор $A \in \mathcal{L}(E_1, E_2)$,
тогда существует естественный способ определить
отображение $A^* : E_2^* \to E_1^*$:
\[
  g \mapsto g \circ A;
\]
композиция сохраняет линейность
и непрерывность, а потому определение корректно.
Более того, очевидно, что данный оператор линеен.

\begin{definition}
  \( A^* \) называется \emph{сопряжённым} к \( A \) оператором.
\end{definition}

В случае гильбертовых пространств, благодаря теореме
Рисса"--~Фреше можно перенести оператор с сопряжённых пространств
на исходные. При этом, такой "<перенесённый"> оператор
будет задаваться уравнением
\[
  \Inner{A x, y}_{H_2} = \Inner{x, A^* y}_{H_1}.
\]

\begin{exercise}
  Доказать это в частном случае \( E_1 = E_2 = H \),
  т. е. если \(H \) "--- гильбертово пространство и
  \( A \in \mathcal{L}(H) \), то
  существует единственный оператор
  \( A^* \in \mathcal{L}(H) \) такой,
  что
  \( \Forall{x, y \in H} \Inner{Ax, y} = \Inner{x, Ay} \).
  Доказать также следующие соотношения
  (\( B \in \Linears{H} \),
  \( \alpha \) и \( \beta \) "--- скаляры):
  \begin{enumerate}
    \item $(\alpha A + \beta B)^* = \overline{\alpha} A^* + \overline{\beta} B^*$,
    \item $I^* = I$,
    \item $(A^*)^* = A$,
    \item $(AB)^* = B^* A^*$.
  \end{enumerate}
\end{exercise}

\begin{theorem}%11.1
  Пусть $E_1$, $E_2$ "--- линейное нормированное пространство,
  $A \in \mathcal{L}(E_1, E_2)$.
  Тогда $A^* \in \mathcal{L}(E_2^*, E_1^*)$
  и $||A^*|| = ||A||$.
\end{theorem}
\begin{proof}
  Как уже было сказано, оператор очевидным образом
  корректен и линеен; осталось разобраться
  с непрерывностью и нормой.
  С одной стороны,
  \[
    |(A^* g)(x)| = |g(Ax)| \le
    ||g|| \cdot ||Ax|| \le ||g|| \cdot ||A|| \cdot ||x||,
  \]
  т. е. $||A^* g|| \le ||A|| \cdot ||g||$.
  Значит, $A^*$ ограничен и, более того,
  $||A^*|| \le ||A||$.

  Осталось доказать, что $||A^*|| \ge ||A||$.
  Действительно, по следствию из теоремы Хана"--~Банаха для произвольного $x \in E_1$
  \[
    ||A x|| = \sup_{||g||_{E_2^*} = 1} |g(Ax)| =
    \sup_{||g|| = 1} |(A^* g)(x)| \le
    %\sup_{||g|| = 1} ||A^* g|| \cdot ||x|| \le
    \sup_{||g|| = 1} \left(||A^*|| \cdot ||g|| \cdot ||x||\right) =
    ||A^*|| \cdot ||x||.
  \]
\end{proof}

% отступление про т. Фредгольма

\begin{theorem}\label{thm:conjugate-complement}%11.2
  Пусть $H$ "--- гильбертово пространство,
  \( A \in \Linears{H} \). Тогда
  \[
    H = \overline{\Img A} \oplus \Ker A^*.
  \]
\end{theorem}
\begin{proof}
  Покажем, что $(\Img A)^\perp = \Ker A^*$:
  \[
    y \in (\Img A)^\perp
    \oTTo \Forall{x \in H} (Ax, y) = 0
    \oTTo \Forall{x \in H} (x, A^* y) = 0
    \oTTo A^* y = 0
    \oTTo y \in \Ker A^*.
  \]
  Мы знаем, что
  \( \overline{\Img A} = (\Img A)^\perp \),
  а значит теорема о проекции
  применённая к \( \overline{\Img A} \)
  даёт нам требуемое тождество.
\end{proof}

%\begin{exercise}
%  $A \in \mathcal{L}(l_2)$,
%  $(Ax)_n = \sum_{k=1}^\infty a_{nk} x_k$,
%  $\sum |a_{nk}|^2 < \infty$.
%  Тогда $A^*$ задаётся $b_{ij} = \overline{a_{ji}}$.
%\end{exercise}

\begin{theorem}[б/д]%11.3
  Пусть \( E_1 \) и \( E_2 \) "--- банаховы пространства,
  \( A \in \Linears{E_1, E_2} \).
  Тогда
  \( \exists A^{-1} \in \Linears{E_2, E_1} \)
  в том и только в том случае,
  когда \( \exists (A^*)^{-1} \in \Linears{E_1^*, E_2^*} \).
  При этом, если эти условия выполнены,
  \( (A^*)^{-1} = (A^{-1})^* \).
\end{theorem}

\begin{exercise}
  Доказать теорему в случае \( E_1 = E_2 = H \).
\end{exercise}

\end{document}
